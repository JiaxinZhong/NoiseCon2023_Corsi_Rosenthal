% #################################################################################
%
% LaTeX Template LaTeX for Noise-Con 2023
%
%
% #################################################################################
\documentclass[letterpaper,12pt]{article}


%% Pick the one corresponding to your system
%\usepackage[latin1]{inputenc}
%\usepackage[ansinew]{inputenc}
\usepackage[utf8x]{inputenc}
\usepackage[T1]{fontenc}
\usepackage{times}
\usepackage[colorlinks=true,linkcolor=black,citecolor=black,urlcolor=blue]{hyperref}
\usepackage{geometry}
\geometry{top=0.9in,bottom=0.9in,left=0.9in,right=0.9in}
\pagestyle{empty}

\usepackage{titlesec}
\titleformat{\section}
{\bfseries\uppercase}{\thesection.}{1em}{}
\titleformat{\subsection}
{\bfseries}{\thesection.\thesubsection.}{1em}{}
\renewcommand{\labelitemi}{\textendash}
\renewcommand{\labelitemii}{\textendash}

\usepackage{graphicx} % used to insert the figure
\usepackage{multirow} % used for the table
%\usepackage[font=it]{caption}
\usepackage{cite}
\usepackage{breakurl}
\usepackage{indentfirst}
\usepackage{amsmath, amssymb, amsfonts, bm}
\usepackage{txfonts}
\usepackage{enumitem}
\usepackage{xcolor}
\usepackage{enumitem}

\hyphenpenalty=10000
\setlength{\emergencystretch}{3em}

\columnsep 1cm
\setlength{\parindent}{0.5cm}

\titlespacing*{\subsection}{0pt}{1.5em}{0.2em}


\renewcommand\eqref[1]{Equation~\ref{#1}}

\renewcommand{\thesection}{\arabic{section}}
\renewcommand{\thesubsection}{\arabic{subsection}}
\renewcommand{\arraystretch}{1.25}


%\renewcommand{\refname}{6. \hspace{3mm} REFERENCES} 
\renewcommand{\refname}{REFERENCES} 
\setlength{\footnotesep}{12pt} 
%Some stuff with item spacing in lists, related to enumitem package.   See individual lists. 
%\setlist[2]{noitemsep}
%\setenumerate{noitemsep}

% Sort out linespacing with references
\newlength{\bibitemsep}\setlength{\bibitemsep}{.2\baselineskip plus .05\baselineskip minus .05\baselineskip}
\newlength{\bibparskip}\setlength{\bibparskip}{0pt}
\let\oldthebibliography\thebibliography
\renewcommand\thebibliography[1]{%
  \oldthebibliography{#1}%
  \setlength{\parskip}{\bibitemsep}%
  \setlength{\itemsep}{\bibparskip}%
}
 
\begin{document}

\begin{center}
    \includegraphics[width=2.65in]{fig/logo_NC23.jpg}
\end{center}
\vskip.5cm

\begin{flushleft}
\fontsize{16}{20}\selectfont\bfseries
\color{red}(The title should be written in "Times New Roman", 16point, bold font. The first letter of the first word in the title should be capitalized)  \\
\color{black}Instructions for preparing the manuscript of the papers for Proceedings of NOISE-CON 2023 GRAND RAPIDS
\end{flushleft}
\vskip1cm

\renewcommand\baselinestretch{1}
\begin{flushleft}

Given Name Family Name1\footnote{mail1@example.com} add co-authors at the same institution\\
Institution\\
Full address\\

\vskip.5cm
Given Name Family Name2\footnote{mail2@example.com}\\
Institution\\
Full address\\

\end{flushleft}


\textbf{\centerline{ABSTRACT}}\\

Corsi-Rosenthal (CR) boxes efficiently and cost-effectively clean the air in an environment; however, the constant broadband noise they produce is a concern. Measuring sound power levels of CR boxes would provide a metric for noise comparisons, but traditional sound power level measurements are not readily adaptable for the average consumer. Measuring the sound power level in a free field over a reflecting plane is an accurate method but is complicated and expensive involving multiple microphones in an anechoic chamber (ANSI S12.54-1999). The Comparison Method is a simpler procedure that can be performed directly in the environment of interest but requires an expensive calibrated source (ANSI S12.57-2011). To allow for simple and inexpensive sound power measurements of CR boxes, we used ANSI S12.54-1999 to measure the sound power of a \$30 hand vacuum (Black+Decker HNVC115JB06) to explore its potential as a DIY calibrated source for the Comparison Method. The reproducibility and uncertainty of this DIY method will be discussed. The DIY method has the potential to enable noise comparison of air cleaners or other devices for the average consumer. 

\section{INTRODUCTION}

\noindent
This paper is a template to which you can refer for your paper.  I suggest that the first step is to compile this document and make sure everything is working on your computer.  Your PDF file should match what you should see with Template.pdf. 

It is anticipated that the NOISE-CON 2023 GRAND RAPIDS Proceedings will be distributed to the congress participants on a memory stick or available for download from a website. 

\begin{itemize}[noitemsep]
\item 
The purpose of these instructions is to ensure the uniformity of the publication. 
\item
The manuscript should be submitted as a PDF file whose font is 12-point "Times New Roman".
\item
The length of a manuscript should be at most 12 pages and at least four pages. 
\item
Only manuscripts in English will be accepted for the Proceedings. 
\item
You must not insert any page number, header or foot note except the e-mail addresses on the first page of the manuscript.
\end{itemize}
 
% \vspace{1in}
 
\section{Manuscript Format}
\noindent
Margin settings, paragraphs, figure and table are explained here. 

\subsection{Margin Settings}
\begin{itemize}[noitemsep]
\item
The paper size is US Letter: 8.5"(21.6cm) wide by 11"(27.9cm) tall.
\item
Margin settings: Top, Bottom, Left, and Right all 0.8"(2cm).
\item
The text should be justified from left to right (i.e., fully justified).
\item
The first line of the paragraphs should be indented by 0.25in (0.5cm).  No indent on paragraphs immediately after a heading or subheading.
\end{itemize}

\subsection{Paragraphs}

\begin{itemize}[noitemsep]
\item
There should be one empty line between headings and subheadings when there is no text between the two types of headings. 
\item
Major headings shall be numerically ordered as 1., 2., ... all in bold font and all upper case.
\item
Level 2 subheading should be 2.1., 2.2., ..., in bold font lower case letters, with upper case letters for the start of major words in the subheading. 
\end{itemize}

\subsection{Figures, Equations, Tables}
\noindent
All figures, tables, equations, photos, graphs, etc., must be shown shortly after they are mentioned, placed at the centre of a page.  \\

The captions of figures (which may include photos) are put below the figures and photos (see Figure \ref{fig:1}).  They are centered if one line or less long, and fully justified if longer than one line.  They should be referred to in the text as Figure 1, Figure 2, etc. 
\begin{figure}[h!]
\begin{center}
  \includegraphics[width=2.65in]{logo_NC23.jpg}
  \end{center}
  \caption{The NOISE-CON 2023 logo.}
  \label{fig:1}
\end{figure}

The equations should be referenced as Equation 1, Equation 2, etc. For example:  Equation \ref{Eq:1}, the formula for estimating the mean value, is:

\begin{equation}
\bar{X} = \frac{1}{N} \sum_{i=1}^{N} X_i ,
\label{Eq:1}
\end{equation}
\noindent
where $X_i$, $=1,2,...N$ denote the $N$ measurements, and $\bar{X}$ denotes the estimated mean value. 
\\

The caption for a table should be placed just above the table and the table number should be Table 1, Table 2,.... Tables should be referred to in the text as, e.g., Table \ref{Tab:1}.

\begin{table}[h!]

\caption{Example of values displayed in a table.}
\label{Tab:1}

\begin{center}
\begin{tabular}{c c c } 
 \hline
 \textbf{Test Number} &  \textbf{Variable 1 (m/s)}& \textbf{Distance (m)}  \\ [0.5ex] 
 \hline
 1 & 6 & 87837 \\ 
 \hline
 2 & 7 & 78  \\
 \hline
 3 & 545 & 778 \\ [1ex] 
 \hline
\end{tabular}
\end{center}
\end{table}
%\vspace{1in}
\section{Referencing other work}
\noindent
Use a numerical referencing system, in order of citation.  For Latex users it should take care of this for you if you use bibtex.   Either put the references in this file or use a .bib file and let Latex pull out the references that you are using.  The .bib file in this example is sample.bib..  The Latex references cited here are:  \cite{latexcompanion, knuthwebsite}.  I have included some additional references as examples for format \cite{Poulsen1, Ryherd2007, Tang2006, May96,  Zwicker_Fastl_2006, ANSI_S3_4}.\ Use the unsrt bibliography style file.  

\section{IMPORTANT Uploading INFORMATION}
\noindent
Here are the instructions for submitting manuscripts. 
\subsection{Submission of Manuscripts}
\noindent
Submit your manuscript as a PDF file using the link on the NOISE-CON 2023 website. 

Before submitting the manuscript, you need to pay the registration fee and if you submit multiple manuscripts, you need to pay the extra nominal charge for each additional manuscript.

\subsection{Conversion to PDF}
\noindent
Before submission, you need to check your PDF file carefully to be sure that PDF creation was done properly and there is no error when the PDF file is opened. The following problems may occur.
\begin{itemize}[noitemsep]
\item
Symbols are missed.
\item
Symbols are converted incorrectly, especially mathematical symbols.
\item
Figures are missed.
\item
Indentation is not correct.
\end{itemize}

\noindent
The author is responsible for sorting out these problems and the manuscript  in the Congress Proceedings will be as it was received.

The name of the file you submit should be: NC\_2023\_YYYY.pdf, where YYYY is your abstract submission number. If your abstract submission number is less than 4 digits add zeros to the front of the number to make it 4 digits.  Some examples are shown in Table \ref{Tab:2}.
\begin{table}[h!]

\caption{Naming convention for PDF files that contain your paper.}
\label{Tab:2}

\begin{center}
\begin{tabular}{c c} 
 \hline
{\textbf{Abstract }} & \\
{\textbf{Submission}} & { \textbf{Filename}}\\
{\textbf{Number}} &\\
 \hline
 12 & NC\_2023\_0012.pdf  \\ 
 \hline
 130 & NC\_2023\_0130.pdf   \\
 \hline
 1985 & NC\_2023\_1985.pdf  \\ [1ex] 
 \hline
\end{tabular}
\end{center}
\end{table}

\section{CONCLUSIONS}

\noindent
Please follow the instructions.  Mostly the Latex stuff will look after you, but check everything is working. Recompile this document to make sure you get the same as the Format\_Description.pdf. 

\section*{Acknowledgements}
\noindent
We gratefully acknowledge the authors for submitting their work to NOISE-CON 2023 GRAND RAPIDS to be held in Grand Rapids, Michigan, USA, May 15-18 2023.

\bibliographystyle{unsrt}
\bibliography{sample} 

\end{document}

